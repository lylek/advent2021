\documentclass{article}
\usepackage{amsmath}
\begin{document}
\title{Advent of Code 2021: Day 7}
\author{Lyle Kopnicky}
\maketitle
\section{Introduction}
The crabs are located at positions $p_i$ for $0 \le i < n$.
The cost of fuel for a crab to move from position $x$ to position $y$ is
\begin{equation}\label{eq:fuel-cost}
  c(x, y) = \sum_{j=0}^{|x-y|} j.
\end{equation}
To minimize the amount of fuel expended by the crabs, we must choose a position $y$
such that the total cost for all crab movements,
\begin{equation}
  t(y) = \sum_{i=0}^{n-1} c(p_i, y),
\end{equation}
is minimized.

By finding the antidifference of \eqref{eq:fuel-cost}, we can see that the
fuel cost is very close to the square of the distance moved:
\begin{equation}
  c(x, y) = \frac{|x - y|(|x - y| + 1)}{2}
\end{equation}

This is very close to the square of the distance.
So, it seems possible that, since the arithmetic mean minimizes
the sum of squared errors, it might minimize the sum
of this fuel cost.

In fact, it does, and we will prove it. However, there's a hitch:
the mean may not be integral. In that case, it turns out that the
position of minimal cost will be one or both of the integers closest
to the mean, on either side.

Our proof consists of two steps:
\begin{enumerate}
  \item Prove that if a position is to the left of the mean, then
  moving further left will increase the total cost.
  \item Prove that if a position is to the right of the mean, then
  moving further right will increase the total cost.
\end{enumerate}
If we can prove these, then we can be sure that there are at most
two points with minimum cost, and if there are two, they are adjacent.
We can also calculate the mean and know that either:
\begin{enumerate}
  \item It falls on an integer: Then the mean is the unique position
        of minimum cost.
  \item It falls between integers: Then one or both of the positions
        to the left or right of the mean has the minimum cost.
\end{enumerate}
So, finding the position with minimum cost would involve calculating
the mean, then testing the adjacent points to see which has a lower
cost.

\section{Left of the mean}

Let's first show that for any position to the left of the mean,
moving further left increases the cost. First, recall the formula
for the mean:
%
\begin{equation}\label{eq:mean}
  m = \frac{1}{n} \sum_{i=0}^{n-1} p_i
\end{equation}

Now choose an arbitrary position $y$ less than or equal to the mean $m$.
We want to show that the total cost is higher at $y-1$ than at $y$.
To do this, we will show that $t(y-1)-t(y) > 0$.

\begin{equation*}
  \begin{split}
    t(y-1)-t(y) &= \sum_{i=0}^{n-1}
    \begin{cases}
      p_i - y, & \quad\text{if}\ p_i < y \\
      p_i - y + 1, & \quad\text{if}\ p_i >= y \\
    \end{cases} \\
    &= -ny + \sum_{i=0}^{n-1}
    \begin{cases}
      p_i, & \quad\text{if}\ p_i < y \\
      p_i + 1, & \quad\text{if}\ p_i >= y \\
    \end{cases} \\
    &= -n(y-m+m) + \sum_{i=0}^{n-1}
    \begin{cases}
      p_i, & \quad\text{if}\ p_i < y \\
      p_i + 1, & \quad\text{if}\ p_i >= y \\
    \end{cases} \\
    &= -n(y-m) -nm + \sum_{i=0}^{n-1}
    \begin{cases}
      p_i, & \quad\text{if}\ p_i < y \\
      p_i + 1, & \quad\text{if}\ p_i >= y \\
    \end{cases} \\
    &= n(m-y) -nm + \sum_{i=0}^{n-1}
    \begin{cases}
      p_i, & \quad\text{if}\ p_i < y \\
      p_i + 1, & \quad\text{if}\ p_i >= y \\
    \end{cases} \\
    &= n(m-y) -\sum_{i=0}^{n-1} p_i + \sum_{i=0}^{n-1}
    \begin{cases}
      p_i, & \quad\text{if}\ p_i < y \\
      p_i + 1, & \quad\text{if}\ p_i >= y \\
    \end{cases} \\
    &= n(m-y) + \sum_{i=0}^{n-1}
    \begin{cases}
      0, & \quad\text{if}\ p_i < y \\
      1, & \quad\text{if}\ p_i >= y \\
    \end{cases} \\
  \end{split}
\end{equation*}

Since $y<=m$, and $n$ must be positive, the term $n(m-y)$ must be
non-negative. The summation term must be positive as long as there
is at least one $i$ such that $p_i >= y$.

To show that there is at least one $p_i >= y$, recall that $y <= m$,
so we only need to show that there is at least one $p_i >= m$.
Then consider the definition of the mean \eqref{eq:mean}. If all
the $p_i$ were less than $m$, then $\sum_{i=0}^{n-1} p_i$ would be
less than $nm$, so $m$ would be less than $m$, a contradiction.
Thus there must be at least one $p_i$ greater than or equal to $m$,
and therefore greater than or equal to $y$.

Since we have at least one $p_i >= y$, the last summation term
in the calculation of $t(y-1) - t(y)$ must be positive, and since
the $n(m-y)$ term is non-negative, we have that $t(y-1) - t(y)$ is
positive. Therefore, the total fuel cost at $t(y-1)$ is greater than
the total fuel cost at $t(y)$, when $y <= m$.

\section{Right of the mean}

Secondly, we must prove that when $y >= m$, the total fuel cost
$t(y+1)$ is greater than the total fuel cost $t(y)$. The steps
are the same as the $y <= m$ case, but with some of the signs flipped:

\begin{equation*}
  \begin{split}
    t(y+1)-t(y) &= \sum_{i=0}^{n-1}
    \begin{cases}
      y - p_i, & \quad\text{if}\ p_i > y \\
      y - p_i + 1, & \quad\text{if}\ p_i <= y \\
    \end{cases} \\
    &= ny + \sum_{i=0}^{n-1}
    \begin{cases}
      -p_i, & \quad\text{if}\ p_i > y \\
      -p_i + 1, & \quad\text{if}\ p_i <= y \\
    \end{cases} \\
    &= n(y-m+m) + \sum_{i=0}^{n-1}
    \begin{cases}
      -p_i, & \quad\text{if}\ p_i > y \\
      -p_i + 1, & \quad\text{if}\ p_i <= y \\
    \end{cases} \\
    &= n(y-m) +nm + \sum_{i=0}^{n-1}
    \begin{cases}
      -p_i, & \quad\text{if}\ p_i > y \\
      -p_i + 1, & \quad\text{if}\ p_i <= y \\
    \end{cases} \\
    &= n(y-m) +\sum_{i=0}^{n-1} p_i + \sum_{i=0}^{n-1}
    \begin{cases}
      -p_i, & \quad\text{if}\ p_i > y \\
      -p_i + 1, & \quad\text{if}\ p_i <= y \\
    \end{cases} \\
    &= n(y-m) + \sum_{i=0}^{n-1}
    \begin{cases}
      0, & \quad\text{if}\ p_i > y \\
      1, & \quad\text{if}\ p_i <= y \\
    \end{cases} \\
  \end{split}
\end{equation*}

Since $y>=m$, and $n$ must be positive, the term $n(y-m)$ must be
non-negative. The summation term must be positive as long as there
is at least one $i$ such that $p_i <= y$.

To show that there is at least one $p_i <= y$, recall that $y >= m$,
so we only need to show that there is at least one $p_i <= m$.
Then consider the definition of the mean \eqref{eq:mean}. If all
the $p_i$ were greater than $m$, then $\sum_{i=0}^{n-1} p_i$ would be
greater than $nm$, so $m$ would be greater than $m$, a contradiction.
Thus there must be at least one $p_i$ less than or equal to $m$,
and therefore less than or equal to $y$.

Since we have at least one $p_i <= y$, the last summation term
in the calculation of $t(y+1) - t(y)$ must be positive, and since
the $n(y-m)$ term is non-negative, we have that $t(y+1) - t(y)$ is
positive. Therefore, the total fuel cost at $t(y+1)$ is greater than
the total fuel cost at $t(y)$, when $y >= m$.

\section{Conclusions}

These two proofs lead us to the following conclusions:
\begin{enumerate}
  \item The position $\lfloor m \rfloor$, the greatest integer less
        than or equal to the mean, must have the minimal cost of all
        positions less than or equal to the mean.
  \item The position $\lceil m \rceil$, the least integer greater than
        or equal to the mean, must have the minimal cost of all positions
        greater than or equal to the mean.
  \item Either position $\lfloor m \rfloor$, $\lceil m \rceil$, or
        both, must have minimal total fuel cost.
  \item If $m$ is integral, then that position has minimal total fuel
        cost.
\end{enumerate}

\end{document}
